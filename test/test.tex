\documentclass{article}

\usepackage[ngerman]{babel}
\usepackage[utf8]{inputenc}
\usepackage[T1]{fontenc}
\usepackage{lmodern}
\usepackage{amsmath,amssymb,amsfonts,amstext,amsthm}
\usepackage[rgb]{xcolor}
\usepackage[author={PDF Tools}]{pdfcomment}
\usepackage{attachfile}

\pagestyle{empty}
\begin{document}
\section{Text}

\label{sec:text}
PDF Tools is, among other things, a replacement of DocView for PDF
files.  The key difference is, that pages are not prerendered by
e.g. ghostscript and stored in the file-system, but rather created
on-demand and stored in memory.

PDF Tools is, among other things, a replacement of DocView for PDF
files.  The key difference is, that pages are not prerendered by
e.g. ghostscript and stored in the file-system, but rather created
on-demand and stored in memory.
\newpage

\section{Annotations}

\label{sec:annotations}
\begin{enumerate}
\item \pdfcomment[color=red,icon=Insert]{insert} \\
\item \pdfmarkupcomment[markup=StrikeOut, color=red]{strikeout} \\
\item \pdfmarkupcomment[markup=Highlight, color=red]{highlight} \\
\item \pdfmarkupcomment[markup=Underline, color=red]{underline} \\
\item \pdfmarkupcomment[markup=Squiggly, color=red]{squiggly} \\
\end{enumerate}
\newpage

\section{Links}

\label{sec:links}

\begin{enumerate}
\item Internal link: \ref{sec:text}
\item External link: \url{http://www.gnu.org}
\end{enumerate}
\newpage

\section{Attachments}

\label{sec:attachments}
\attachfile{test.tex}

\newpage
\section{Outline}

\label{sec:outline}
\subsection{subsection}
\subsubsection{subsubsection}

\newpage
\indent
% empty page
\end{document}
